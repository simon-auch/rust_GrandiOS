\documentclass[a4paper,12pt]{article}
\usepackage[utf8]{inputenc}
\usepackage{amsfonts,amstext}
\usepackage{amsmath}
\usepackage{german}
\usepackage{todonotes}

\begin{document}
\todo[inline]{title}

\section*{Aufgabe 1}
(Paper: Towards Real \(\mu\)-Kernels (1996))
\todo[inline]{2.Paper}

\todo[inline]{Aufgabe ist bzgl makro nicht monolithisch \(\rightarrow\) einordnen usw.}
\(\mu\)-Kernel im Bezug zu monolithischen

\begin{itemize}
  \item nur IPC, MMU, Scheduler Teil des Kernels
  \item Interrupts werden ausserhalb des Kernels behandelt
  \item[+] nur Kernel kann sicherheitskritische Operationen eines Prozessors nutzen, software in user mode nicht
  \item[+] Treiberausfälle etc. sind \emph{nur} Softwarefehler
  \item[+] modularer/flexibel/leicht erweiterbar da nicht Kernel für neue Geräte angepasst/erweitert werden muss
  \item[+] Kernel besser wartbar, da kleiner
  \item[+] Treiber etc. nur Zugriff auf zugewiesenen Speicherbereichen
  \item[-] ineffizienter / mehr overhead bei IPCs, Addressraum wechsel etc.
  \item[-] Je nach Hardwarezugriff können nun (leicht austauschbare) Treiber (im user mode) das System korrumpieren
\end{itemize}

\end{document}
