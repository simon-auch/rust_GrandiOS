\documentclass[a4paper,12pt]{article}
\usepackage[utf8]{inputenc}
\usepackage{amsfonts,amstext}
\usepackage{amsmath}
\usepackage{german}
\usepackage{todonotes}

\begin{document}
\HEADER{3}{Betriebssysteme}{WiSe 2017}{Sebastian Börner, Christian Hofmann, Simon Auch}{7 Dezember 2017, 10 Uhr}
\begin{description}
\AUFGABE{Nachrichten unterschiedlicher Länge}{5 Punkte}

Im Allgemeinen sind an Kommunikationsobjekten Nachrichten unterschiedlicher Länge zugelassen.
    \begin{itemize}
      \item Welche Auswirkungen hat dies auf die Daten des Kommunikationsobjekts? Bei welchen Varianten der Nachrichtenübergabe hat dies weitere Auswirkungen, und bei welchen hat dies sonst keine weiteren Auswirkungen?
      \item Welche Probleme können aus der Sicht der Kommunikationspartner auftreten, wenn Nachrichten unterschiedlicher Länge zugelassen sind? Schlagen Sie geeignete Lösungen vor!
    \end{itemize}

\textbf{Lösung:}
    \begin{itemize}
      \item Daten unterschiedlicher Länge sind genau dann kein Problem, wenn trotz der Unterschiedlichkeit ein Maximum existiert (z.B. Enums, in historischen Sprachen Unions genannt). Man geht einfach bei Empfänger- und Versender-Seite davon aus, dass jede Nachricht diese Maximallänge einnimmt. Die Information der konkreten Ausprägung ist entweder ohnehin Teil dieses Datums oder kann schon auf Versender-Seite nicht festgestellt werden und muss daher auch nicht separat kodiert werden.
        Auch stellt sich die Frage der übertragungsfreundlichen Darstellung, wofür allerdings üblicherweise entweder direkt der Speicher so übertragen wird, wie er auf der Versender-Seite im Speicher liegt (eventuelle Anpassungen bei verschiedener Endianness notwendig), oder man verwendet eine geeignete Form der Serialisierung.

        Weitere Auswirkungen haben Nachrichten unterschiedlicher Länge lediglich bei theoretisch unbeschränkten (oder zu schwach beschränkten, denn ein Array hat theoretisch auch eine Maximallänge) Nachrichten.
        Dies sind im Wesentlichen folgende:
      \item Bei Nachrichten unterschiedlicher Länge entsteht auf Seiten des Empfängers das Problem, dass möglicherweise für die Nachricht veranschlagter Speicher überschritten und in andere Speicherbereiche geschrieben wird, oder ein Re-Alloziieren und damit ein Kopieren des gesamten bis dahin empfangenen Inhalts notwendig ist -- unter Umständen mehrfach für eine Nachricht.

        Es gibt 2 mögliche Lösungsansätze für diese Problematik:
        \begin{itemize}
          \item Das Einsetzen einer Datenstruktur, die in $\mathcal{O}(1)$ erweitert werden kann (z.B. Linked List), dies hat aber zum Einen unter Umständen erheblichen Daten-Overhead (und vielen Alloc-Aufrufen), da zum Übertragen von beispielsweise eines Strings mit 8-Byte-Elementen jeweils ein Pointer auf den Nachfolger (hier  Bytes) gespeichert werden müsste und insgesamt der 5-fache Speicher der Nachrichtengröße benötigt wird, zum Anderen sind unter Umständen erstrebenswerte Operationen nicht oder nicht einfach so möglich (Index-Zugriff bei Linked Lists liegt in $\mathcal{O}(n)$) und erfordern ein abschließendes Umwandeln der zum Empfang benutzten Datenstruktur in ein entsprechend geeigneteres Format.

            Diese Probleme lassen sich teilweise verringern, indem Hybridisierung mit der Ziel-Datenstruktur stattfindet, beispielsweise eine Linked List, die nicht einzelne Zeichen sondern Zeichen-Arrays einer maximalen Länge speichert, man steht hier allerdings vor einem ähnlichen Problem wie bei der Frage nach interner vs externer Fragmentierung.
          \item Falls der Empfänger bereits weiß, wie lang die zu sendenden Daten sind, kann er diese Länge einfach der Nachricht voranstellen, sodass der Empfänger in der Lage ist, exakt die benötigte Menge Speicher zu reservieren, wie für diese Nachricht notwendig ist. Diese Lösung hat viele Vorteile wie beispielsweise die Möglichkeit des direkten Verwendens der Zieldatenstruktur (z.B. Array), einmaliger Alloc-Aufruf und keinerlei Daten-Overhead.

            Nachteile sind, dass auf theoretischer Ebene unklar sein könnte, wie viel der Nachricht die Längeninformation einnimmt. Wenn die zu sendenden Daten beispielsweise mehr als 2\^32 Byte groß sind und für die Längeninformation nur 32 Bit Größe vorgesehen sind, lassen sich diese Daten nur über Umwege auf diesem Weg versenden; in der Praxis ist dies allerdings vernachlässigbar, da die Größe der Adressen ohnehin eine Maximalgröße vorgibt. Auch problematisch sind Daten mit unbekannter Länge wie beispielsweise Streams. Diese ließen sich mit der erstgenannten Methode problemlos versenden, allerdings benötigt eine Mischform beider Verfahren zusätzlich das Metadatum, welche Form der Übertragung nun konkret stattfindet. Auch entsteht bei Streaming-Algorithmen -- also solchen, die gar nicht alle Daten zur gleichen Zeit benötigen -- mit dieser Methode ein enormer Daten-Overhead, der unter Umständen gar nicht behandelt werden kann.
        \end{itemize}
    \end{itemize}

\AUFGABE{Kontextwechsel und präemptives Multitasking}{15 Punkte}

In  dieser  Aufgabe  soll  Ihr  Betriebssystem  um  die  Fähigkeit  des  präemptiven  Multitaskings erweitert  werden,  um  so  mehrere  Threads  quasi-parallel  ausführen  zu  können.  Der  System-Timer dient  dabei  als  Zeitgeber  und  gibt  das  Intervall  vor,  in  dem  der  Kern  den  gerade ausgeführten Thread automatisch wechselt. 

\textbf{Lösung:}
Anleitung zum Erstellen eines mit qemu benutzbaren Kernels:
\begin{enumerate}
  \item Wechseln in den Ordner \texttt{GrandiOS}
  \item Zunächst wird mit dem Script \texttt{setup\_env.sh} die Toolchain von Rust vorbereitet. Unter Umständen sind Benutzereingaben erwartet, diese können einfach mit Enter bestätigt werden.
  \item Ausführen von \texttt{make run} zum Erstellen eines Kernels und starten von Qemu mit diesem.
\end{enumerate}
Das Vorbereiten der Toolchain mittels \texttt{setup\_env.sh} ist nur einmal notwendig.

Für den unwahrscheinlichen Fall, dass aus irgendeinem Grund das Bauen des Kernels fehlschlägt, ist unter {\texttt{GrandiOS/kernels/kernel\_04}} noch ein von uns gebauter Kernel vorhanden.

Eine kleine Übersicht, wo die für diese Aufgabe relevanten Codesegmente zu finden sind:
\begin{itemize}
	\item \texttt{GrandiOS/src/utils/scheduler.rs} Implementation der Thread-Verwaltung. Unsere Zeitscheibenlänge beträgt standardmäßig 2 Sekunden, es sei denn ein Interrupt verkürzt diese. Insbesondere werden jedoch auch bei Benutzung eines Sleep-SWI die Zeitscheibe so eingestellt, dass der nächste Thread pünktlich gestartet werden kann. Falls es mehrere Threads der gleichen Priorität gibt, ergibt sich durch die Verwendung der Priority-Queue automatisch ein Round-Robin-Scheduling. Die maximale/minimale Zeitscheibenlänge kann hierbei zu Beginn/am Ende der Switch-Funktion eingestellt werden (in Ticks, wobei ein Tick als 1/1024 Sekunden eingestellt ist). Der Fall, dass es keinen Thread zum Ausführen gibt, kann nicht eintreten, da wir immer einen Idle-Thread haben, welcher immer ausgeführt werden kann (und nicht beendet werden kann). Der Idle-Thread pausiert insbesondere auch die CPU (siehe ganz unten in der Datei). Ein Thread kann mittels des Exit-Syscalls beendet werden, dies entfernt ihn aus allen Warteschlangen aus dem Scheduler und gibt später vielleicht irgendwann mal benutzte Ressourcen frei.
	\item \texttt{GrandiOS/src/utils/exceptions/common\_code.rs} Beinhaltet den Trampolincode für unsere Interrupts (sowohl swi als auch irq). Hierbei werden alle relevanten Register auf dem Interrupt-Stack gesichert. Falls der Thread gewechselt werden soll müssen diese nur woanders gespeichert werden und im Stack überschrieben werden.
	\item \texttt{GrandiOS/src/utils/exceptions/irq.rs} Beinhaltet die Implementierung des Threads, den wir starten, als auch die Ausgabe des Ausrufezeichens. Der Thread benutzt leider kein Sleep-Aufruf, da er hierfür in der aktuellen implementation einen Allocator benutzen muss, was jedoch im Kontext des Kernels (zu dem diese Datei gehört) leider der gleiche Allocator wie vom Scheduler ist und daher nicht ohne Kernellock benutzt werden sollte. Deswegen machen wir stattdessen einfach busy-waiting.
\end{itemize}

\end{description}
\end{document}
